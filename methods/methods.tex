\documentclass[12pt]{article}
\usepackage{amsmath}
\usepackage{bbm}
\usepackage{microtype}
\usepackage{upgreek}
\usepackage[hidelinks]{hyperref}
\usepackage{titlesec}
%usepackage{helvet}
%\renewcommand{\familydefault}{\sfdefault}
%usepackage{setspace}
%\linespread{1.25}
\titlespacing{\section}{0pt}{0pt}{-.5em}
\usepackage[top=1in,bottom=1in,right=1in,left=1in]{geometry}
\setlength{\parindent}{0em}
\setlength{\parskip}{1em}
\newcommand{\ifna}{IFN\hspace{-.08em}${\upalpha 2}$}
\newcommand{\ifnb}{IFN\hspace{-.03em}${\upbeta}$}
\newcommand{\icFifty}{IC$_{50}$}
\begin{document}
\begin{center}
 \Large\textbf{}
\end{center}
\section*{Bayesian models of IFN-I resistance in plasma isolates from longitudinally sampled participants}
To create a simple model of the temporal dynamics of type I interferons (IFN-I) \ifna{} and \ifnb{} resistance, \icFifty{} values were each modeled using a Bayesian change point hierarchical model.
  The model is based on a segmented regression of the log \icFifty{} making the following assumptions:
    \begin{itemize}
          \item Each participant has a level of resistance at the acute infection stage drawn from separate population-level distributions for typical, non- or fast progressors.
          \item Each participant has a drop (or rise) in IFN-I resistance from acute levels drawn from separate population-level distributions for typical, non- or fast progressors.
          \item Each participant has a time to nadir drawn from a population-level distribution shared among all progression types.
          \item Resistance changes linearly from onset of symptoms to time of nadir.
          \item After the nadir, the level of IFN-I resistance changes linearly with fluctuations in CD4+ T cell counts away from the level found at nadir, with the effect for each participant drawn from separate population-level distributions for typical, non- or fast progressors.
          \item CD4+ T cell counts were linearly interpolated between observations.
    \end{itemize}
  
 The log \icFifty{} observation from each viral isolate $i$ was modeled as a normal distribution $\text{IC50}_i\sim\text{Normal}(\mu_i,\sigma)$ with mean $\mu_i$ where:
\begin{align*}
  \mu_i=\begin{cases} 
    \alpha_{\text{person}_i}+\delta_{\text{person}_i}\times\frac{\text{time}_i}{s_{\text{person}_i}} & \text{if } \text{time}_i<s_{\text{person}_i}\\
    \alpha_{\text{person}_i}+\delta_{\text{person}_i} + \beta_{\text{person}_i}\times(\text{CD4}_{\text{person$_i$},\text{time}_i}-\text{CD4}_{\text{person$_i$},s_\text{person$_i$}}) & \text{if } \text{time}_i\ge s_{\text{person}_i}\\
  \end{cases}
\end{align*}
where the parameters $\alpha_j$ represent the level of IFN-I resistance at symptom onset, $\delta_j$ represents the change from symptom onset to nadir and $s_j$ represents the time of nadir in person $j$. Study participant data is represented by $\text{time}_i$ corresponding to the time since onset of symptoms and $\text{person}_i$ recording the participant from which isolate $i$ was collected, $\text{CD4}_{j,k}$ containing the estimated CD4+ T cell count for person $j$ at time $k$ and $\text{progression}_j$ is the disease progression type (fast/non/typical) for participant $j$. The hierarchical probabilities for these parameters were:
\begin{align*}
  \sigma & \sim \text{Gamma}(1,0.1)\\
  \alpha_j& \sim \begin{cases}
    \text{Normal}(\theta_{\alpha,\text{typical}},\tau_{\alpha})  &\text{ if } \text{progression}_j=\text{typical}\\
    \text{Normal}(\theta_{\alpha,\text{typical}}+\theta_{\alpha,\text{fast}},\tau_{\alpha}) &\text{ if } \text{progression}_j=\text{fast}\\
    \text{Normal}(\theta_{\alpha,\text{typical}}+\theta_{\alpha,\text{non}},\tau_{\alpha}) &\text{ if } \text{progression}_j=\text{non}\\
  \end{cases}\\
  \delta_j& \sim \begin{cases}
    \text{Normal}(\theta_{\delta,\text{typical}},\tau_{\delta})  &\text{ if } \text{progression}_j=\text{typical}\\
    \text{Normal}(\theta_{\delta,\text{typical}}+\theta_{\delta,\text{fast}},\tau_{\delta}) &\text{ if } \text{progression}_j=\text{fast}\\
    \text{Normal}(\theta_{\delta,\text{typical}}+\theta_{\delta,\text{non}},\tau_{\delta}) &\text{ if } \text{progression}_j=\text{non}\\
  \end{cases}\\
  s_j & \sim \text{NegativeBinomial}(\theta_s,\tau_s)\\
  \beta_j& \sim \begin{cases}
    \text{Normal}(\theta_{\beta,\text{typical}},\tau_{\beta})  &\text{ if } \text{progression}_j=\text{typical}\\
    \text{Normal}(\theta_{\beta,\text{fast}},\tau_{\beta})  &\text{ if } \text{progression}_j=\text{fast}\\
    \text{Normal}(\theta_{\beta,\text{non}},\tau_{\beta})  &\text{ if } \text{progression}_j=\text{non}\\
  \end{cases}\\
\end{align*}
where $j$ indicates each participant and $\text{NegativeBinomial(x,y)}$ represents a negative binomial distribution parameterized such that the expected value is $x$ and the variance is $x+\frac{x^2}{y}$. All hyperparameters were given prior probabilities of $\theta_x \sim \text{Normal(0,10)}$ for parameters representing the means of a distribution and $\tau_x \sim \text{Gamma}(1,0.1)$ for parameters representing standard deviations other than $\theta_{\alpha,\text{typical}}$ and $\theta_s$ which were given a flat prior and $\tau_s \sim \text{Cauchy}(0,10)$.

For computational efficiency, the nadir time parameter $s$ was discretized to weekly intervals, assumed to fall within 1 to 150 weeks after symptom onset and marginalized out of the joint probability:
\[p(\text{IC50},\text{...})=p(\text{...})\prod_{i=1}^n\sum_{s=1}^{150}\text{Normal}(\text{IC50}_i|\mu_{i,s},\sigma)\text{NegativeBinomial}(s|\theta_s,\tau_s)\]
where $\text{...}$ represents all parameters other than $s$ and $\mu_{i,s}$ is defined the same as $\mu_i$:
\[\mu_{i,s}=\begin{cases}
    \alpha_{\text{person}_i}+\frac{\text{time}_i}{s}\delta_{\text{person}_i} & \text{if } \text{time}_i<s\\
    \alpha_{\text{person}_i}+\delta_{\text{person}_i} + \beta_{\text{person}_i}(\text{CD4}_{\text{person$_i$},\text{time}_i}-\text{CD4}_{\text{person$_i$},s}) & \text{if } \text{time}_i\ge s\\
  \end{cases}
\]


\section*{Bayesian models of IFN-I resistance of outgrowth and rebound isolates}
  To compare the IFN-I resistance of viral isolates derived from plasma samples collected during acute, chronic and rebound infections as well as from viably frozen PBMCs collected during ART suppression (QVOA), \ifna{} and \ifnb{} \icFifty{} values were modeled using a Bayesian hierarchical model. The model is based on the assumptions that:
  \begin{itemize}
    \item Isolates found at  acute infection form a base level of IFN-I resistance for a given person. Virus isolated from chronic, rebound and QVOA for this person are modelled as changes from this initial level.
    \item The mean \icFifty{} level within each person for a given virus type (acute, chronic, rebound, QVOA) was assumed to be drawn from a population-level distribution for that type. 
    \item QVOA virus are separated into to two populations; a ``pre'' group composed of QVOA viruses isolated from study participants prior to or in the absence of treatment interruption (ATI) and a ``post'' group of QVOA viruses isolated from participants following ATI and reinitiation of ART.
    \item In both QVOA populations, the viruses can include some proportion of rebound-like isolates. This mixture is modeled in both pre- and post-treatment so that differences in mixture proportion between the two populations can be assessed.
    \item  Batch-to-batch variation in IFN-I potency may need correction between isolates tested in Iyer et al. 2017 and the current study.
    \item Isolates from patients treated with exogenous \ifna{} may differ from those not receiving such treatment.
  \end{itemize}

  The log \icFifty{} observation from each viral isolate $i$ from acute, chronic and rebound isolates was modeled as a normal distribution:
  \[\text{IC50}_i\sim\text{Normal}(\mu_{\text{type}_i,\text{person}_i},\sigma_\text{type$_j$})\]
  with the mean resistance for isolate type $j$ from person $k$:
  \begin{align*}
    \mu_{j,k}\sim&\begin{cases}
      \text{Normal}(\alpha_{k}+\beta_{\text{batch}}\text{batch}_k,\psi_j) & \text{if } j=\text{acute}\\
      \text{Normal}(\alpha_{k}+\beta_{j,k}+\beta_{\text{batch}}\text{batch}_k+\beta_{\text{\textsc{ifn}}}\text{IFN}_{k},\psi_j) & \text{if } j=\text{rebound}\\
      \text{Normal}(\alpha_{k}+\beta_{j,k}+\beta_{\text{batch}}\text{batch}_k,\psi_j) & \text{otherwise}
    \end{cases}
  \end{align*}
  where $\text{type}_i$ indicates whether isolate $i$ was isolated during acute, chronic, QVOA or rebound infection from participant $\text{person}_i$, $\text{batch}_k$ indicates when isolates from person $k$ were tested in Iyer et al. 2017 (to account for batch variation in IFN potency) and $\text{IFN}_k$ indicates when person $k$ was treated with exogenous \ifna{} prior to and during treatment interruption. Parameters are included for the mean resistance level during acute infection for each person $\alpha_k$, standard deviation of isolates of type $j$ within a person $\sigma_j$, standard deviation of mean resistance for type $j$ isolates among people $\psi_j$, change from acute levels in isolates of type $j$ in a given participant $\beta_{j,k}$, the effects of exogenous IFN treatment $\beta_{\textsc{ifn}}$ and batch to batch variation in IFN in isolates previously assayed by Iyer et al. 2017 $\beta_{\text{batch}}$. 


  For QVOA isolates, the \icFifty{} was modeled as a mixture of two populations such that:
  \[
  \begin{split}
    p(\text{IC50}_i|\mu_{\text{\textsc{qvoa}},\text{person}_i},&\sigma_{\text{\textsc{qvoa}}},\mu_{\text{rebound},\text{person}_i},\sigma_{\text{rebound}},\phi_{\text{prePost}_i})=\\
    &\phi_{\text{prePost}_i}\text{Normal}(\text{IC50}_i|\mu_{\text{rebound},\text{person}_i},\sigma_{\text{rebound}})\\
    &+ (1-\phi_{\text{prePost}_i})\text{Normal}(\text{IC50}_i|\mu_{\text{\textsc{qvoa}},\text{person}_i},\sigma_{\text{\textsc{qvoa}}}) 
  \end{split}
  \]
  where $\text{prePost}_i$ indicates whether isolate $i$ was isolated  pre- or post-ATI and and $\phi_\text{pre}$ and $\phi_\text{post}$ represent the proportion of rebound-like virus present in pre- and post-ATI QVOA isolates.


The hierarchical parameter priors were modeled as:
  \begin{align*}
    \sigma_j & \sim \text{Gamma}(1,0.1)\\
    \psi_j & \sim \text{Gamma}(1,0.1)\\
    \phi_{\text{pre}} & \sim \text{Uniform}(0,1)\\
    \phi_{\text{post}} & \sim \text{Uniform}(0,1)\\
  \end{align*}
where $j$ indicates the isolate type (acute, chronic, QVOA, rebound). $\alpha_k$, $\beta_{j,k}$, $\beta_{\text{\textsc{ifn}}}$ and $\beta_{\text{batch}}$ were all given flat priors.


\end{document}

